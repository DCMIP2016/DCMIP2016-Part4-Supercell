%% Copernicus Publications Manuscript Preparation Template for LaTeX Submissions
%% ---------------------------------
%% This template should be used for copernicus.cls
%% The class file and some style files are bundled in the Copernicus Latex Package which can be downloaded from the different journal webpages.
%% For further assistance please contact the Copernicus Publications at: publications@copernicus.org
%% http://publications.copernicus.org


%% Please use the following documentclass and Journal Abbreviations for Discussion Papers and Final Revised Papers.


%% 2-Column Papers and Discussion Papers
\documentclass[gmd, manuscript]{copernicus}



%% Journal Abbreviations (Please use the same for Discussion Papers and Final Revised Papers)

% Archives Animal Breeding (aab)
% Atmospheric Chemistry and Physics (acp)
% Advances in Geosciences (adgeo)
% Advances in Statistical Climatology, Meteorology and Oceanography (ascmo)
% Annales Geophysicae (angeo)
% ASTRA Proceedings (ap)
% Atmospheric Measurement Techniques (amt)
% Advances in Radio Science (ars)
% Advances in Science and Research (asr)
% Biogeosciences (bg)
% Climate of the Past (cp)
% Drinking Water Engineering and Science (dwes)
% Earth System Dynamics (esd)
% Earth Surface Dynamics (esurf)
% Earth System Science Data (essd)
% Fossil Record (fr)
% Geographica Helvetica (gh)
% Geoscientific Instrumentation, Methods and Data Systems (gi)
% Geoscientific Model Development (gmd)
% Geothermal Energy Science (gtes)
% Hydrology and Earth System Sciences (hess)
% History of Geo- and Space Sciences (hgss)
% Journal of Sensors and Sensor Systems (jsss)
% Mechanical Sciences (ms)
% Natural Hazards and Earth System Sciences (nhess)
% Nonlinear Processes in Geophysics (npg)
% Ocean Science (os)
% Proceedings of the International Association of Hydrological Sciences (piahs)
% Primate Biology (pb)
% Scientific Drilling (sd)
% SOIL (soil)
% Solid Earth (se)
% The Cryosphere (tc)
% Web Ecology (we)
% Wind Energy Science (wes)


%% \usepackage commands included in the copernicus.cls:
%\usepackage[german, english]{babel}
%\usepackage{tabularx}
%\usepackage{cancel}
%\usepackage{multirow}
%\usepackage{supertabular}
%\usepackage{algorithmic}
%\usepackage{algorithm}
%\usepackage{amsthm}
%\usepackage{float}
%\usepackage{subfig}
%\usepackage{rotating}

% Custom commands
\newcommand{\vb}{\mathbf}
\newcommand{\vg}{\boldsymbol}
\newcommand{\mat}{\mathsf}
\newcommand{\diff}[2]{\frac{d #1}{d #2}}
\newcommand{\diffsq}[2]{\frac{d^2 #1}{{d #2}^2}}
\newcommand{\pdiff}[2]{\frac{\partial #1}{\partial #2}}
\newcommand{\pdiffsq}[2]{\frac{\partial^2 #1}{{\partial #2}^2}}


\begin{document}

\title{DCMIP2016, Part 4: Splitting Supercell}


% \Author[affil]{given_name}{surname}

\Author[1]{Colin}{Zarzycki}
\Author[2]{Christiane}{Jablonowski}
\Author[3]{James}{Kent}
\Author[1]{Peter}{Lauritzen}
\Author[1]{Ramachandran}{Nair}
\Author[4]{Kevin A.}{Reed}
\Author[5]{Paul A.}{Ullrich}

\Author[6]{Thomas}{Dubos}
\Author[7]{Marco}{Giorgetta}
\Author[8]{Elijah}{Goodfriend}
\Author[9]{David A.}{Hall}
\Author[10]{Lucas}{Harris}
\Author[8]{Hans}{Johansen}
\Author[11]{Christian}{Kuehnlein}
\Author[12]{Vivian}{Lee}
\Author[13]{Thomas}{Melvin}
\Author[14]{Hiroaki}{Miura}
\Author[15]{David}{Randall}
\Author[16]{Alex}{Reinecke}
\Author[1]{William}{Skamarock}
\Author[16]{Kevin}{Viner}
\Author[17]{Robert}{Walko}

\affil[1]{National Center for Atmospheric Research}
\affil[2]{University of Michigan}
\affil[3]{University of South Wales}
\affil[4]{Stony Brook University}
\affil[5]{University of California, Davis}
\affil[6]{Institut Pierre-Simon Laplace (IPSL)}
\affil[7]{Max Planck Institute for Meteorology}
\affil[8]{Lawrence Berkeley National Laboratory}
\affil[9]{University of Colorado, Boulder}
\affil[10]{Geophysical Fluid Dynamics Laboratory}
\affil[11]{European Center for Medium-Range Weather Forecasting}
\affil[12]{Environment Canada}
\affil[13]{U.K. Met Office}
\affil[14]{University of Tokyo}
\affil[15]{Colorado State University}
\affil[16]{Naval Research Laboratory}
\affil[17]{University of Miami}

%% The [] brackets identify the author with the corresponding affiliation. 1, 2, 3, etc. should be inserted.



\runningtitle{DCMIP2016: Splitting Supercell}

\runningauthor{Zarzycki, et al.}

\correspondence{Colin Zarzycki (zarzycki@ucar.edu)}



\received{}
\pubdiscuss{} %% only important for two-stage journals
\revised{}
\accepted{}
\published{}

%% These dates will be inserted by Copernicus Publications during the typesetting process.


\firstpage{1}

\maketitle



\begin{abstract}
This paper discusses a new idealized test for atmospheric dynamical cores.
\end{abstract}



\introduction  %% \introduction[modified heading if necessary]



The supercell test permits the study of a non-hydrostatic moist feature with strong vertical velocities and associated precipitation and is based on the work of \cite{klemp1978simulation}.

\begin{table}[h]

\caption{List of constants used for the Supercell test}

\begin{tabular*}{\textwidth}{@{\extracolsep{\fill}}lll}
\hline Constant & Value & Description \\
\hline
$X$ & $120$ & Small-planet scaling factor (reduced Earth)\\
$\theta_{tr}$ & $343$ K & Temperature at the tropopause \\
$\theta_0$ & $300$ K & Temperature at the equatorial surface \\
$z_{tr}$ & $12000$ m & Altitude of the tropopause \\
$T_{tr}$ & $213$ K & Temperature at the tropopause \\
$U_s$ & $30$ m/s & Maximum zonal wind velocity \\
$U_c$ & $15$ m/s & Coordinate reference velocity \\
$z_{s}$ & $5000$ m & Lower altitude of maximum velocity \\
$\Delta z_{u}$ & $1000$ m & Transition distance of velocity \\
$\Delta \theta$ & $3$ K & Thermal perturbation magnitude \\
$\lambda_p$ & $0$ & Thermal perturbation longitude \\
$\varphi_p$ & $0$ & Thermal perturbation latitude \\
$r_p$ & $X \times 10000$ m & Perturbation horizontal half-width \\
$z_{c}$ & $1500$ m & Perturbation center altitude \\
$z_{p}$ & $1500$ m & Perturbation vertical half-width \\
\hline 
\end{tabular*}

\end{table}

It is assumed that the saturation mixing ratio is given by
\begin{equation}
q_{vs}(p,T) = \left( \frac{380.0}{p} \right) \exp\left(17.27 \times \frac{T-273.0}{T-36.0}\right)
\end {equation}  The definition of this test case relies on hydrostatic and gradient wind balance, written in terms of Exner pressure $\pi$ and virtual potential temperature $\theta_v$ as
\begin{equation} \label{eq:BalanceEq}
\pdiff{\pi}{z} = - \frac{g}{c_p \theta_v}, \quad \mbox{and} \quad u^2 \tan \varphi = - c_p \theta_v \pdiff{\pi}{\varphi}.
\end{equation}  These equations can be combined to eliminate $\pi$, leading to
\begin{equation} \label{eq:CombinedBalanceEq}
\pdiff{\theta_v}{\varphi} = \frac{\sin (2 \varphi)}{2 g} \left( u^2 \pdiff{\theta_v}{z} - \theta_v \pdiff{u^2}{z} \right).
\end{equation}

The wind velocity is analytically defined throughout the domain.  Meridional and vertical wind is initially set to zero.  The zonal wind is obtained from
\begin{equation}
\overline{u}(\varphi,z) = \left\{ \begin{array}{ll}
\displaystyle \left(U_s\frac{z}{z_t}-U_c\right)\cos(\varphi) & \mbox{for $z < z_s - \Delta z_u$}, \\[2.0ex]
\displaystyle \left(-\frac{4}{5}+3\frac{z}{z_s}-\frac{5}{4}\frac{z^2}{z_s^2}\right)U_s-U_c & \mbox{for $\vert z-z_s \vert \leq \Delta z_u$} \\[2.0ex]
\displaystyle \left(U_s-U_c\right)\cos(\varphi) & \mbox{for $z > z_s + \Delta z_u$}
\end{array} \right.
\end{equation}

The equatorial profile is determined through numerical iteration.  Potential temperature at the equator is specified via
\begin{equation}
\theta_{\text{eq}}(z) = \left\{ \begin{array}{ll} \displaystyle \theta_0 + (\theta_{tr} - \theta_0)\left(\frac{z}{z_{tr}}\right)^{\frac{5}{4}}  & \mbox{for $0 \leq z \leq z_{tr}$,} \\[2.0ex]
\displaystyle \theta_{tr} \exp\left(-\frac{g(z-z_{tr})}{c_pT_{tr}}\right) & \mbox{for $z_{tr} \leq z$} \end{array} \right.
\end{equation}  And relative humidity is given by
\begin{equation}
\overline{H}(z) = \left\{ \begin{array}{ll} \displaystyle 1 + \frac{3}{4}\left(\frac{z}{z_{tr}}\right)^{5/4} & \mbox{for $0 \leq z \leq z_{tr}$,} \\[2.0ex]
\displaystyle \frac{1}{4} & \mbox{for $z_{tr} \leq z$.} \end{array} \right.
\end{equation}  Pressure and temperature at the equator are obtained by iterating on hydrostatic balance  with initial state
\begin{equation}
\theta_{v,\text{eq}}^{(0)}(z) = \theta_{\text{eq}}(z),
\end{equation} and iteration procedure
\begin{align}
\pi_{\text{eq}}^{(i)} &= 1 - \int_{0}^{z} \frac{g}{c_p \theta_{v,\text{eq}}^{(i)}} dz \\
p_{\text{eq}}^{(i)} &= p_0 (\pi^{(i)})^{c_p / R_d} \\
T_{\text{eq}}^{(i)} &= \theta_{\text{eq}}(z) \pi_{\text{eq}}^{(i)} \\
q^{(i)}_{\text{eq}} &= H(z) q_{vs}(p_{\text{eq}}^{(i)}, T_{\text{eq}}^{(i)}) \\
\theta_{v,\text{eq}}^{(i+1)} &= \theta_{\text{eq}}(z) (1 + M_v q^{(i)}_{\text{eq}})
\end{align}  This iteration procedure appears to converge to machine epsilon after approximately 10 iterations.  The equatorial moisture profile is then extended through the entire domain,
\begin{equation}
q(z, \varphi) = q_{\text{eq}}(z).
\end{equation}

Once the equatorial profile has been constructed, the virtual potential temperature through the remainder of the domain can be computed by iterating on (\ref{eq:CombinedBalanceEq}),
\begin{align}
\theta_v^{(i+1)}(z,\varphi) = \theta_{v,\text{eq}}(z) + \int_{0}^{\varphi} \frac{\sin(2 \phi)}{2 g} \left(\overline{u}^2 \pdiff{\theta_{v}^{(i)}}{z} - \theta_v^{(i)} \pdiff{\overline{u}^2}{z} \right) d\varphi.
\end{align}  Again, approximately 10 iterations are needed for convergence to machine epsilon.  Once virtual potential temperature has been computed throughout the domain, Exner pressure throughout the domain can be obtained from (\ref{eq:BalanceEq}),
\begin{equation}
\pi(z,\varphi) = \pi_{eq}(z) - \int_{0}^{\varphi} \frac{u^2 \tan \varphi}{c_p \theta_v} d\varphi,
\end{equation} and so
\begin{align}
p(z,\varphi) &= p_0 \pi(z,\varphi)^{c_p / R_d}, \\
T_v(z,\varphi) &= \theta_v(z,\varphi) (p / p_0)^{R_d / c_p}.
\end{align}

\subsection{Potential temperature perturbation}

To initiate convection, a thermal perturbation is introduced in the initial potential temperature field:
\begin{equation}
\theta^\prime(\lambda,\phi,z) = \left\{ \begin{array}{ll} \displaystyle \Delta \theta \cos^2\left(\frac{\pi}{2}R_{\theta}(\lambda, \varphi, z)\right) & \mbox{for $R_{\theta}(\lambda, \varphi, z) < 1$,} \\[2.0ex]
0 & \mbox{for $R_{\theta}(\lambda, \varphi, z) \geq 1$,} \end{array} \right.
\end{equation} where
\begin{equation}
R_{\theta}(\lambda, \varphi, z) = \left[ \left( \frac{R_c(\lambda, \varphi; \lambda_p, \varphi_p)}{r_p} \right)^2 + \left( \frac{z - z_c}{z_p} \right)^2 \right]^{1/2}.
\end{equation}

\ \\
\noindent \begin{tabular}{|p{\textwidth}|}
\hline \textbf{Note:} An additional iterative step will be required here to bring the potential temperature perturbation into hydrostatic balance.  Without this additional iteration, large vertical velocities will be generated as the model rapidly adjusts to hydrostatic balance. \\
\hline
\end{tabular}

\begin{figure}[tb]
\center\includegraphics[width=\linewidth]{plot_supercell_init.pdf}
  \caption{Initial state for the supercell test.}\label{fig:supercell_init_p1}
\end{figure} 
\begin{figure}[tb]
\center\includegraphics[width=\linewidth]{plot_supercell_init_p2.pdf}
  \caption{Initial state for the supercell test (cont'd).}\label{fig:supercell_init_p2}
\end{figure} 

\subsection{Grid spacings, simulation time, output and diagnostics}

\begin{itemize}
\item Moist simulations should be performed at 1$^\circ$ resolution with 30 vertical levels for 2 hours.
\item Plots of vertical velocity and rainwater should be produced at 5 km altitude after 30, 60, 90 and 120 minutes over the domain $[0, 130E] \times [40S, 40N]$.
\item A plot of maximum vertical velocity over the duration of the simulation should be produced.
\item Experiments could address the coupling frequency between the dynamics and physics.
\item A variable resolution simulation should be performed that (a) studies the effect of the supercell transitioning from fine resolution to coarse resolution and (b) high resolution simulations down to 0.125$^\circ$ over the supercell.
\end{itemize}


\conclusions  %% \conclusions[modified heading if necessary]
TEXT



%%%%%%%%%%%%%%%%%%%%%%%%%%%%%%%%%%%%%%%%%%%%%%%%%%%%%%%%%%%%%

\appendix

\section{Uniform Diffusion} \label{sec:UniformDiffusion}


%\subsection{}                               %% Appendix A1, A2, etc.


\authorcontribution{TEXT}

\begin{acknowledgements}
{\color{blue}[Include a complete list of DCMIP2016 student participants here along with sponsors]}
\end{acknowledgements}


%% REFERENCES

%% Since the Copernicus LaTeX package includes the BibTeX style file copernicus.bst,
%% authors experienced with BibTeX only have to include the following two lines:
%%
\bibliographystyle{copernicus}
\bibliography{DCMIP2016-Part4}
%%
%% URLs and DOIs can be entered in your BibTeX file as:
%%
%% URL = {http://www.xyz.org/~jones/idx_g.htm}
%% DOI = {10.5194/xyz}


%% LITERATURE CITATIONS
%%
%% command                        & example result
%% \citet{jones90}|               & Jones et al. (1990)
%% \citep{jones90}|               & (Jones et al., 1990)
%% \citep{jones90,jones93}|       & (Jones et al., 1990, 1993)
%% \citep[p.~32]{jones90}|        & (Jones et al., 1990, p.~32)
%% \citep[e.g.,][]{jones90}|      & (e.g., Jones et al., 1990)
%% \citep[e.g.,][p.~32]{jones90}| & (e.g., Jones et al., 1990, p.~32)
%% \citeauthor{jones90}|          & Jones et al.
%% \citeyear{jones90}|            & 1990



%% FIGURES

%% ONE-COLUMN FIGURES

%%f
%\begin{figure}[t]
%\includegraphics[width=8.3cm]{FILE NAME}
%\caption{TEXT}
%\end{figure}
%
%%% TWO-COLUMN FIGURES
%
%%f
%\begin{figure*}[t]
%\includegraphics[width=12cm]{FILE NAME}
%\caption{TEXT}
%\end{figure*}
%
%
%%% TABLES
%%%
%%% The different columns must be seperated with a & command and should
%%% end with \\ to identify the column brake.
%
%%% ONE-COLUMN TABLE
%
%%t
%\begin{table}[t]
%\caption{TEXT}
%\begin{tabular}{column = lcr}
%\tophline
%
%\middlehline
%
%\bottomhline
%\end{tabular}
%\belowtable{} % Table Footnotes
%\end{table}
%
%%% TWO-COLUMN TABLE
%
%%t
%\begin{table*}[t]
%\caption{TEXT}
%\begin{tabular}{column = lcr}
%\tophline
%
%\middlehline
%
%\bottomhline
%\end{tabular}
%\belowtable{} % Table Footnotes
%\end{table*}
%
%
%%% NUMBERING OF FIGURES AND TABLES
%%%
%%% If figures and tables must be numbered 1a, 1b, etc. the following command
%%% should be inserted before the begin{} command.
%
%\addtocounter{figure}{-1}\renewcommand{\thefigure}{\arabic{figure}a}
%
%
%%% MATHEMATICAL EXPRESSIONS
%
%%% All papers typeset by Copernicus Publications follow the math typesetting regulations
%%% given by the IUPAC Green Book (IUPAC: Quantities, Units and Symbols in Physical Chemistry,
%%% 2nd Edn., Blackwell Science, available at: http://old.iupac.org/publications/books/gbook/green_book_2ed.pdf, 1993).
%%%
%%% Physical quantities/variables are typeset in italic font (t for time, T for Temperature)
%%% Indices which are not defined are typeset in italic font (x, y, z, a, b, c)
%%% Items/objects which are defined are typeset in roman font (Car A, Car B)
%%% Descriptions/specifications which are defined by itself are typeset in roman font (abs, rel, ref, tot, net, ice)
%%% Abbreviations from 2 letters are typeset in roman font (RH, LAI)
%%% Vectors are identified in bold italic font using \vec{x}
%%% Matrices are identified in bold roman font
%%% Multiplication signs are typeset using the LaTeX commands \times (for vector products, grids, and exponential notations) or \cdot
%%% The character * should not be applied as mutliplication sign
%
%
%%% EQUATIONS
%
%%% Single-row equation
%
%\begin{equation}
%
%\end{equation}
%
%%% Multiline equation
%
%\begin{align}
%& 3 + 5 = 8\\
%& 3 + 5 = 8\\
%& 3 + 5 = 8
%\end{align}
%
%
%%% MATRICES
%
%\begin{matrix}
%x & y & z\\
%x & y & z\\
%x & y & z\\
%\end{matrix}
%
%
%%% ALGORITHM
%
%\begin{algorithm}
%\caption{�}
%\label{a1}
%\begin{algorithmic}
%�
%\end{algorithmic}
%\end{algorithm}
%
%
%%% CHEMICAL FORMULAS AND REACTIONS
%
%%% For formulas embedded in the text, please use \chem{}
%
%%% The reaction environment creates labels including the letter R, i.e. (R1), (R2), etc.
%
%\begin{reaction}
%%% \rightarrow should be used for normal (one-way) chemical reactions
%%% \rightleftharpoons should be used for equilibria
%%% \leftrightarrow should be used for resonance structures
%\end{reaction}
%
%
%%% PHYSICAL UNITS
%%%
%%% Please use \unit{} and apply the exponential notation


\end{document}
